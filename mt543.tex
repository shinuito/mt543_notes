\documentclass[12pt,a4paper]{article}
%packages
\usepackage[utf8]{inputenc}
\usepackage{amsmath}
\usepackage{amsfonts}
\usepackage{amssymb}
\usepackage{amsthm}
\usepackage{physics}
\usepackage{tikz-cd}

\usepackage{chngcntr}
\counterwithout{subsection}{section}
\setcounter{subsection}{-1}

\title{MT543 Topics in Algebra}
\author{Notes taken by Stephen Nulty}
\renewcommand{\abstractname}{Note:} 

%new commands
\newcommand{\rR}{\ensuremath{\mathbb{R}\,}}
\newcommand{\cC}{\ensuremath{\mathbb{C}\,}}
\newcommand{\hH}{\ensuremath{\mathbb{H}\,}}
\newcommand{\qQ}{\ensuremath{a+bi+cj+dk\,}}
\newcommand{\nn}{\ensuremath{n \times n\,}}

\newcommand{\mnr}{\ensuremath{M_n(\rR)\,}}
\newcommand{\mnc}{\ensuremath{M_n(\cC)\,}}
\newcommand{\mnh}{\ensuremath{M_n(\hH)\,}}
\newcommand{\mr}[1]{\ensuremath{M_{#1}(\rR)\,}}
\newcommand{\mc}[1]{\ensuremath{M_{#1}(\cC)\,}}
\newcommand{\mh}[1]{\ensuremath{M_{#1}(\hH)\,}}
\newcommand{\glnr}{\ensuremath{GL_n(\rR)\,}}
\newcommand{\glnc}{\ensuremath{GL_n(\cC)\,}}
\newcommand{\glnh}{\ensuremath{GL_n(\hH)\,}}
\newcommand{\glr}[1]{\ensuremath{GL_{#1}(\rR)\,}}
\newcommand{\glc}[1]{\ensuremath{GL_{#1}(\cC)\,}}
\newcommand{\glh}[1]{\ensuremath{GL_{#1}(\hH)\,}}

\newcommand{\ra}{\Rightarrow}
\newcommand{\la}{\Leftarrow}
\newcommand{\cin}{\ensuremath{\mathcal{I}_n\,}}
\newcommand{\ci}[1]{\ensuremath{\mathcal{I}_{#1}\,}}

%proofs etc
\newtheorem{thm}{Theorem}
\newtheorem{defn}[thm]{Definition}
\newtheorem{lemma}[thm]{Lemma}
\newtheorem{prop}[thm]{Proposition}
\newtheorem{obs}[thm]{Observation}
\newtheorem{cor}[thm]{Corollary}

%block matrices
\newcommand{\rvline}{\hspace*{-\arraycolsep}\vline\hspace*{-\arraycolsep}}
\newcommand{\bigzero}{\mbox{\normalfont\Large\bfseries 0}}
\newcommand{\bigld}{\mbox{\normalfont\Large\bfseries \ldots}}
\newcommand{\bigvd}{\mbox{\normalfont\Large\bfseries \vdots}}

\renewcommand{\thesubsection}{\arabic{subsection}}
\makeatletter
\def\@seccntformat#1{\@ifundefined{#1@cntformat}%
   {\csname the#1\endcsname\quad}%    default
   {\csname #1@cntformat\endcsname}}% enable individual control
\newcommand\section@cntformat{}     % section level 
\newcommand\subsection@cntformat{\thesubsection.\space} % subsection level
\newcommand\subsubsection@cntformat{\thesubsubsection.\space} % subsubsection level
\makeatother

\renewcommand{\thethm}{\arabic{subsection}.\arabic{subsubsection}.\arabic{thm}}

%begin
\begin{document}
\maketitle
\begin{abstract}
    Any transcription mistakes and typos are my own.
\end{abstract}
Lectures by David Wraith. Lie Groups and Lie Algebras.
\section{Lecture 1 25/09/23}
missed this lecture - some intro to do with spheres, transformations and symmetries and other motivational stuff. Definition of an algebra (bilinear product) over a field.

\subsection{Something}

\section{Lecture 2 27/09/23}
Sorting out tutorial times. Lectures: Monday 2pm MS2, Wednesday 2pm LGH, Thursday 12pm MS2.

Lie Groups, dual nature, Groups but also a topological geometrical character. Can prove things with a mix of both methods - intersection of various areas.

\subsection{Groups of matrices}
\subsubsection{General Linear Groups}

Quaternions will have a central role.

Consider groups of $N\times N$ matrices over the fields \rR and \cC and also over the quaternions.

\begin{defn}
The quaternions \hH is a 4-dim real vector space with standard basis elements $1,i,j,k$, equipped with an associative linear multiplication operation defined by

\[i^2=j^2=k^2=-1, \quad ij=k, jk=i, ki=j\]
\end{defn}

So a generic quaternion takes the form $a+bi+cj+dk$, $a,b,c,d \in \rR$.

Observe, $ji=j(jk)=(jj)k$ (by associativity) $=j^2k=-k$. Similarly $kj=-i$ and $ik=-j$.

e.g. $(2+i-3k)(5+2i-j+k)=10+4i-2j+2k+5i-2-k-j-15k-6j-61+3$ etc.

Quaternions is not commutative, so is not a field. However it is a skew field (division algebra).

Terminology - In $a+bi+cj+dk$, $a$ is called the real or scalar part, and the rest $bi+cj+dk$ imaginary or vector part.

In analogy with complex numbers,

\begin{defn} 
1. The conjugate of  \qQ , is $\overline{\qQ}=a-bi-cj-dk$
2. The norm of \qQ is  $|\qQ |=\sqrt{a^2+b^2+c^2+d^2}$
\end{defn}

Thus \hH is a normed vector space. Next observe that for each $q\in\hH$ $q\bar{q}=\bar{q}q = |q|^2$.

therefore (symbol) $q^{-1}=\bar{q}/|q|^2$. So $qq^{-1}=q\bar{q}/|q|^2=|q|^2=1$, similarly for $q^{-1}q=1$.

This allows division $q_1\cdot q_2^{-1}=q_1 \bar{q_2}/|q_2|^2$. Writing $q_1/q_2$ is ambiguous however. $q_1 q_2^{-1}\neq q_2^{-1} q_1$ generically.

Clearly $\rR \subset \cC \subset \hH$. A classic theorem of Frobenius asserts that \rR, \cC, \hH are the only real associative division algebras. These objects similarly play a distinguished role in Lie group theory.

Convention: Suppose V is a vector space over the quaternions \hH. We will adopt the convention that whenever we scale a vector $v\in V$ by a scalar $\lambda \in \hH$, we multiply on the left, i.e. $\lambda v$

Let $M_n(\rR),  M_n(\cC),  M_n(\hH)$ denote the sets (vector spaces!) of all $n\times n$ matrices over $\rR, \cC,  \hH$.

\begin{defn} 
The General Linear Groups $GL_n(\rR)$, resp. $GL_n(\cC)$ is the group of  $n\times n$ invertible matrices with \rR resp \cC coefficients. (Group under multiplication). Equivalently $GL_n(\rR)=\{A\in M_n(\rR)| \det(A)\neq0\}$. Similarly $GL_n(\cC)=\{A\in M_n(\cC)| \det(A)\neq0\}$.
\end{defn}

(return to the idea of determinants of quaternions later).

Recall that for any matrix $A\in\mnr$ we have two associated linear maps $L_A:\rR^n\to \rR^n, L_a(\vec{x})=A\vec{x},$  $R_A:\rR^n\to \rR^n, R_a(\vec{x})=\vec{x}A.$ 

It is well know that $A$ is invertible (RC cases) $\iff$ $\det(A)\neq0$ $\iff$ $L_A, R_A$ are isomorphisms.

\section{Lecture 3 02/10/23}
Thursday lecture moved to Friday at 10am in MS2.

Reminder: 
\begin{itemize}
\item Quaternions \hH, multiplication is associative not commutative. If $V$ is a \hH - vector space, we scale from the left only, i.e. $\lambda v$ for $\lambda \in \hH, v\in V$. 
\item General linear groups \glnr, \glnc groups under $*$ of all invertible \rR resp. \cC \nn - matrices.
\item $A\in \mnr, \mnc$ is invertible iff $\det A \neq 0$ iff $L_A, R_A$ are both invertible where $L_a(\vec{x})=A\vec{x}$, $R_a(\vec{x})=\vec{x}A.$ 
\end{itemize}

We now consider \mnh.

\begin{defn}
A function $f:\hH ^n \to \hH^n$ is \hH - linear if  $f ( \lambda _1 v_1+\lambda _2 v_2)=\lambda _1 f(v_1)+\lambda _2 f(v_2),\; \forall \lambda _1 \lambda _2 \in \hH , v_1,v_2 \in \hH ^n$.
\end{defn}

\begin{lemma}
For $A \in \mnh$, $R_A:\hH^n\to \hH^n$ given by $R_a(\vec{x})=\vec{x}A$  for $v\in \hH^n$ a row vector, is \hH - linear, however $L_A$ is in general not \hH - linear.
Proof: exercise
\end{lemma}

idea is that associativity makes $\lambda v A$ ok, but not with left multiplication which is interfered by commutativity.

\begin{lemma}
For $A \in \mnh$, $R_A:\hH^n\to \hH^n$, is an \hH - linear isomorphism iff $A$ is invertible, i.e. $\exists B\in \mnh$ such that $AB=BA=I_n$.
\end{lemma}

\begin{proof}
($\ra$) If $R_A$ is an iso. then there is a \hH -linear inverse $(R_A)^{-1}: \hH ^n \to \hH ^n $. There is a corresponding matrix $B\in \mnh$. Since $R_A \circ (R_A)^{-1}=R_A \circ (R_A)^{-1}=I_n$. we deduce $BA=AB=I_n$ (NB order of matrices here!). Therefore $B=A^{-1}$.

($\la$) Similar.
\end{proof}

\begin{defn}
The quaternionic general linear group $\glnh = \{A\in \mnh | A \text{ is invertible}\} = \{A\in \mnh| R_a \text{ is an iso.}\}$
\end{defn}

NB: There is a problem with the notion of \hH - determinant due to non-commutativity we'll return to this later (possible to define determinant and gl as ones with non-zero determinant, but defining it requires some thought.)

It turns out that we can view \cC and \hH- matrices/linear maps in terms of \rR- matrices. 

\begin{prop}
There is a real linear map $\rho_n: \mnc \to \mr{2n}$ such that the following diagram commutes. 
\begin{center}
\begin{tikzcd}
  \cC^n \arrow{d}{R_A} \arrow{r}{\theta_n}
    & \rR^{2n} \arrow{d}{R_{\rho_n(A)}} \\
  \cC^n \arrow{r}{\theta_n}
&\rR^{2n} \end{tikzcd}
\end{center}
where $\theta _n : \cC ^n \to \rR ^{2n}$ is given by $\theta _n (a_1+ib_1, \ldots , a_n+ib_n )=(a_1,b_1,\ldots, a_n,b_n)$.
\end{prop}

(compactly every complex matrix can be viewed as a real matrix of twice the size)

Remark: $\theta _n $ is  a real linear isomorphism. This forces $R_{\rho _n (A)}=\theta _n \circ R_A \circ \theta _n^{-1}$.

This is linear and therefore there is a corresponding matrix $\in \mr{2n}$.

\begin{proof}
See moodle.
\end{proof}

\begin{obs}
$\rho_n$ is injective. Proof: exercise.
\end{obs}

\begin{lemma}
$\rho _n$ satisfies $\rho_n(AB)= \rho_n(A)\rho_n(B)$. So $\rho_n$ is  an injective real-algebra homomorphism.
\end{lemma}
\begin{proof}
We compose commutative squares from 1.1.8 to get

\begin{center}
\begin{tikzcd}
  \cC^n \arrow{d}{R_A} \arrow{r}{\theta_n}
    & \rR^{2n} \arrow{d}{R_{\rho_n(A)}} \\
  \cC^n \arrow{r}{\theta_n}  \arrow{d}{R_B}
&\rR^{2n} \arrow{d}{R_{\rho_n(B)}} \\
  \cC^n \arrow{r}{\theta_n}
&\rR^{2n} 

\end{tikzcd}
\end{center}

On L.H.S. we have $R_B \circ R_A = R_{AB}$. (note order)

On R.H.S we have $R_{\rho_n(B)} \circ R_{\rho_n(A)} =R_{\rho_n(A)\rho_n(B)}$.

But since LHS is $R_{AB}$ this means $R_{\rho_n(AB)}= \text{composition on RHS}=R_{\rho_n(A)\rho_n(B)}$.
\end{proof}

It's not surjective however. Q: What exactly is $\rho_n(A)$? Consider $(a+ib)\in \mc{1}$. 

\[R_{(a+ib)}(x+iy)=(x+iy)(a+ib)=(ax-by)+i(ay+bx)\]

Now $\theta_1(x+iy)=(x,y) \in \rR^2$ etc.

So $\theta_1((ax-by)+i(ay+bx))=(ax-by,ay+bx)$

The corresponding map from $\rR ^2\to \rR^2$ is $(x,y)\mapsto (ax-by,ay+bx)$. Observe that 

\[\begin{pmatrix} x & y\end{pmatrix}\begin{pmatrix} a & b \\ -b & a \end{pmatrix}=(ax-by,ay+bx)\]

So $\begin{pmatrix} a & b \\ -b & a \end{pmatrix}\in \mr{2}$ corresponds under $\rho_1$ to $(a+ib)\in \mc{1}$.

More generally 

\[\begin{pmatrix}
  a_{11}+ib_{11} & \ldots &   a_{1n}+ib_{1n}\\
  \vdots & \vdots & \vdots \\
   a_{n1}+ib_{n1} & \ldots &   a_{nn}+ib_{nn}\\
\end{pmatrix} \in \mnc\]

corresponds to

\[
\begin{pmatrix}
  \begin{matrix}
  a_{11} & b_{11}  \\
  -b_{11} &   a_{11}
  \end{matrix}
  & \rvline & \bigld  & \rvline &
  \begin{matrix}
  a_{1n} & b_{1n}  \\
  -b_{1n} &   a_{1n}
  \end{matrix}  
  \\
\hline

  \bigvd
  & \rvline & \bigvd  & \rvline &
  \bigvd
  \\
\hline
    \begin{matrix}
  a_{n1} & b_{n1}  \\
  -b_{n1} &   a_{n1}
  \end{matrix}
  & \rvline & \bigld  & \rvline &
  \begin{matrix}
  a_{nn} & b_{nn}  \\
  -b_{nn} &   a_{nn}
  \end{matrix}  
\end{pmatrix} \in \mr{2n}
\]

 is obtained by replacing each \cC entry by its corresponding $2\times 2$ real block.

\section{Lecture 4 04/10/23}
Last time:

\begin{itemize}
\item $A\in \mnh$ then $R_A:\hH^n\to\hH^n$ given by $R_a(\vec{x})=\vec{x}A.$  is \hH - linear (assuming coefficients in \hH multiply on vectors from the left, x row vector). Left multiplication is not in general \hH linear.
\item Under the real linear isomorphism $\theta_n: \cC^n \to \rR^{2n},$ $\theta _n (a_1+ib_1, \ldots , a_n+ib_n )=(a_1,b_1,\ldots, a_n,b_n)$. Any complex-linear map $\cC^n\to \cC^n$ corresponds to a real-linear map $\rR^{2n}\to \rR^{2n}$ and in terms of matrices (an right multiplication) $A\in \mnc$ corresponds to some matrix $\rho_n(A)\in \mr{2n}$.
\item \[\text{If } A=\begin{pmatrix}
  a_{11}+ib_{11} & \ldots &   a_{1n}+ib_{1n}\\
  \vdots & \vdots & \vdots \\
   a_{n1}+ib_{n1} & \ldots &   a_{nn}+ib_{nn}\\
\end{pmatrix}\]
then 
\[
\rho_n(A)=\begin{pmatrix}
  \begin{matrix}
  a_{11} & b_{11}  \\
  -b_{11} &   a_{11}
  \end{matrix}
  & \rvline & \bigld  & \rvline &
  \begin{matrix}
  a_{1n} & b_{1n}  \\
  -b_{1n} &   a_{1n}
  \end{matrix}  
  \\
\hline

  \bigvd
  & \rvline & \bigvd  & \rvline &
  \bigvd
  \\
\hline
    \begin{matrix}
  a_{n1} & b_{n1}  \\
  -b_{n1} &   a_{n1}
  \end{matrix}
  & \rvline & \bigld  & \rvline &
  \begin{matrix}
  a_{nn} & b_{nn}  \\
  -b_{nn} &   a_{nn}
  \end{matrix}  
\end{pmatrix}
\]
\end{itemize}

Consider the \cC linear map $\cC^n\to \cC^n$ given by $z\to zi$. This is $R_A$ where 

\[A= \begin{pmatrix}
  i & \ldots &   0\\
  \vdots & \ddots & \vdots \\
   0 & \ldots &   i\\
\end{pmatrix} =iI\]

For this matrix we have 
\[\rho_n(A)=\begin{pmatrix}
  \begin{matrix}
  0 & 1  \\
  -1 &   0
  \end{matrix}
  & \rvline & \bigld  & \rvline &
  \bigzero
  \\
\hline

  \bigvd
  & \rvline & \bigvd  & \rvline &
  \bigvd
  \\
\hline
    \bigzero
  & \rvline & \bigld  & \rvline &
  \begin{matrix}
  0 & 1  \\
  -1 &   0
  \end{matrix}  
\end{pmatrix}=\cin
\]

A map $f:\cC^n\to \cC^n$ is \cC linear if it is real linear and $f(zi)=f(z)i$.

Let $bar{f}:\rR^{2n}\to \rR^{2n}$ be the corresponding \rR linear map and suppose this has matrix $B\in \mr{2n}$. Then the complex linearity requirement is $R_B\circ R_{\cin}=R_{\cin}R_{B}$. 

Since $R_X=R_Y \iff X=Y$ we see this is equivalent to asking $B\cin=\cin B$. i.e. $B\in \mr{2n}$ corresponds under $\theta_n$ to a complex linear map $\iff$  $B\cin=\cin B$.

We'd proved

\begin{cor}
The image of $\rho_n:\mnc \to \mr{2n}$ is the set of all of all matrices in \mr{2n} which commute with \cin .
\end{cor}

Remark: This shows that $\rho_n$ is not surjective.

\begin{lemma}
There is an injective group homomorphism $\rho_n : \glnc \to \glr{2n}$, given by restricting $\rho_n : \mnc \to \mr{2n}$.
\end{lemma}

\begin{proof}
We just have to check that if $A\in \glnc$, then $\rho_n(A)$ is invertible. Clearly $\rho_n(AA^{-1})=\rho_n(A^{-1}A)=\rho_n(I_n)$ so by 1.1.10. $\rho_n(A)\rho_n(A^{-1})=\rho_n(A^{-1})\rho_n(A)=\rho_n(I_n)=I_{2n}$. 

$\therefore \rho_n(A^{-1})=\rho_n(A)^{-1}$, hence $\rho_n(A)\in\glr{2n}$. So $\rho_n:\glnc\to\glr{2n}$, and by 1.1.10 this is a (multiplicative) group homomorphism 
\end{proof}

Now for quaternion matrices. 

First observe that there is a \cC linear isomorphism $\phi_n: \hH^n \to \cC^{2n}$ given by $\phi_n(z_1+w_1 j, \ldots, z_n+w_n j)=(z_1,w_1,\ldots,z_n, w_n).$

(exercise to figure out $a+bi+cj+dk$ as  $z+wj$, with $\rR\subset \cC \subset \hH$.)

\begin{prop}
There is an injective \cC linear map $\psi_n:\mnh \to \mc{2n}$ s.t. the following square commutes:

\begin{center}
\begin{tikzcd}
  \hH^n \arrow{d}{R_A} \arrow{r}{\phi_n}
    & \cC^{2n} \arrow{d}{R_{\psi_n(A)}} \\
  \hH^n \arrow{r}{\phi_n}
&\cC^{2n} \end{tikzcd}
\end{center}

i.e. $\phi_n\circ R_A = R_{\psi_n(A)}\circ \phi_n$. Moreover, $\psi_n$ satisfies $\psi_n(AB)=\psi_n(A)\psi_n(B)$.
\end{prop}

\begin{proof}
Analogous to that of prop 1.1.8 and lemma 1.1.10. Exercise! 
\end{proof}

Remark: It is easily checked (exercise!) that $\psi_1(z+wj) = \begin{pmatrix} z &w \\ -\bar{w}& \bar{z}\end{pmatrix}$

More generally, image of $\psi_n$ consists of block matrices with blocks of this form (analogous to $\rho_n$).

By restricting to invertible matrices we obtain:

\begin{cor}
There is an injective group homomorphism $\psi_n:\glnh \to \glc{2n}$.
\end{cor}
\begin{proof}
Analogous to 1.1.12 - exercise.
\end{proof}

( you can compose the maps then to get a real $4n$ matrix from a quaternionic one)

Composing $\rho_{2n}$ and $\psi_n$ gives

 \begin{cor}
There is an injective \rR linear map resp. group homomorphism given by $\rho_{2n}\circ \psi_n:\mnh \to \mr{4n}$ resp. $\rho_{2n}\circ \psi_n:\glnh \to \glr{4n}$.
\end{cor}

Slogan: all groups of \hH or \cC matrices can be viewed as groups of real matrices!

\begin{defn}
For $A\in \mnh$, $\det(A):=\det \psi_n(A)$.
\end{defn}

\end{document}